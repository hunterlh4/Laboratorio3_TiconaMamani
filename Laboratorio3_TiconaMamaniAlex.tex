\documentclass[12pt,a4paper]{book}
 \usepackage[utf8]{inputenc}
\usepackage[spanish]{babel}
\usepackage{amsmath} 
\usepackage{amsfonts}
\usepackage{amssymb}
\usepackage{graphicx}
\usepackage[left=2cm,right=2.5cm,top=2.5cm,bottom=2.5cm]{geometry} \begin{document}
\thispagestyle{empty}

\newcommand{\HRule}{\rule{\linewidth}{0.5mm}}


 {\centering
  
 \begin{figure}[htb] \centering

\includegraphics[scale=1]{Imagenes/upt.jpg}
 \end{figure}


\large{\bf UNIVERSIDAD PRIVADA DE TACNA}\\ \vspace{0.5cm}
\large{\bf FACULTAD DE INGENIERIA }\\ \vspace{0.5cm}

\large{\bf Escuela Profesional de Ingeniería de Sistemas } \\ \vspace{1cm}

{\large {\bf Laboratorio 3 - Unidad I }}\\ \vspace{1cm}

{\large {\bf “POWER BI” }}\\ \vspace{2cm} 

\large Curso: INTELIGENCIA DE NEGOCIOS\\ \vspace{1.5cm}
\large Docente: Ing. Patrick Cuadros Quiroga \\ \vspace{1.5cm}
\large Ticona Mamani, Alex Armando (2017057860) \\ \vspace{4cm}
\vspace{0.5cm} {\Large {\bf \textsc{Tacna - Perú} }}\\ {\Large {\bf \textsc{2019}}} \\}



\begin{center}



\section*{ \\ PRACTICA DE LABORATORIO 03: \\ Creando un Reporte Interactivo en Power BI } 
\end{center}

\hfill \break

\subsubsection*{1.	REQUERIMIENTOS}

\begin{itemize}
\item Conocimientos \\

Para el desarrollo de esta práctica se requerirá de los siguientes conocimientos básicos:

\begin{itemize}
\item Conocimientos básicos de administración de base de datos Microsoft SQL Server.
\item Conocimientos básicos de SQL.

\end{itemize}

\item software 
Asimismo, se necesita los siguientes aplicativos:

\begin{itemize}

\item Microsoft SQL Server 2016 o superior
\item	Base de datos AdventureWorksLT2016 o superior
\item	Tener los archivos de recursos del laboratorio.
\item	Power BI Desktop.
\item	Tener una cuenta Microsoft registrada en el Portal de Power Bi

\end{itemize}

\end{itemize}


\subsubsection*{crear una app}
\begin{itemize}
\item Inicie Qlik Sense.
\item Cree una nueva aplicación.
\item Nombre la aplicación: " Mi análisis de ventas "
\item Crear y abrir la aplicación 
\end{itemize}



\subsubsection*{2.	CONSIDERACIONES INICIALES}

Generar una carpeta o directorio Power BI en un lugar accesible para guardar los resultados de la práctica.


\hfill \break 
\hfill \break
\hfill \break
\hfill \break
\begin{center}
\section*{ \\Ejercicio 1: Conectando a Power BI a datos } 
\end{center}

\subsubsection*{Tarea 1: Conectar a datos existentes}

1. Abrir SQL Server Management Studio, y conectar a la instancia de base de datos (local) utilizando autenticación de Windows.\\
2. En el menú Archivo (File), en el submenú Abrir (Open), hacer click en Project/Solution, y buscar el archivo \\ Project.ssmssln.\\
\begin{center}
\includegraphics[width=8cm]{Imagenes/img12.png}\\
\includegraphics[width=6cm]{Imagenes/img13.png}\\
\end{center}
3. En el Explorador de Soluciones, expandir Consultas (Queries), y luego hacer doble click en el archivo Lab Exercise 1.sql.\\
\begin{center}
\includegraphics[width=6cm]{Imagenes/img14.png}
\end{center}
\begin{center}
\includegraphics[width=10cm]{Imagenes/img15.png} \\
\end{center} 
4. Abrir Power BI Desktop.\\
\begin{center}
\includegraphics[width=12cm]{Imagenes/img16.png}
\end{center}

•	En la ventana Power BI Desktop, hacer click en Obtener Data (Get Data).

\begin{center}
\includegraphics[width=12cm]{Imagenes/img17.png}
\end{center}
\begin{center}
\includegraphics[width=6cm]{Imagenes/img18.png}
\end{center}

5. En la ventana Power BI Desktop, hacer click en Obtener Data (Get Data).\\

\begin{center}
\includegraphics[width=6cm]{Imagenes/img19.png}
\end{center}

\begin{center}
\includegraphics[width=6cm]{Imagenes/img20.png}
\end{center}
6.	En la ventana base de datos Server database, En Servidor, escribir (local). \\
7.	En Base de Datos (opcional), tipear AdventureWorksLT.\\
8.	Expandir el cuadro Opciones Avanzadas. Copiar el script Task 1 del archivo Lab Exercise 1.sql. y pegar la consulta en Power BI, en el cuadro sentencia SQL. Luego presionar OK.

\begin{center}
\includegraphics[width=12cm]{Imagenes/img21.png}
\end{center}
9.	En la ventana de vista preliminar click en Cargar.
\begin{center}
\includegraphics[width=12cm]{Imagenes/img22.png}
\end{center}
10.	En Power BI Desktop, click Obtener Datos y luego click en Mas.
\begin{center}
\includegraphics[width=10cm]{Imagenes/img23.png}
\end{center}
11.	Repetir los pasos del 6 al 10, utilizando el script Task 2.
\begin{center}
\includegraphics[width=12cm]{Imagenes/img24.png}
\end{center}
\begin{center}
\includegraphics[width=12cm]{Imagenes/img25.png}
\end{center}
12.	De regreso en el reporte. Guardar el archivo como AdventureWorksLT Sales.pbix.
\begin{center}
\includegraphics[width=6cm]{Imagenes/img26.png}
\end{center}
\begin{center}
\includegraphics[width=6cm]{Imagenes/img27.png}
\end{center}
\subsubsection*{Tarea 2: Graficar Datos\\}
1. En el panel Campos (Fields), click derecho sobre Query1, Renombrar, tipear Customers y presionar Enter.\\
2. Para el Query2, hacer lo mismo del paso 1 y colocar el nombre Sales.
\begin{center}
\includegraphics[width=6cm]{Imagenes/img28.png}
\end{center}
3.Expandir ambas tablas para ver todas las filas.
\begin{center}
\includegraphics[width=6cm]{Imagenes/img29.png}
\end{center}
4. En la barra de navegación, click Datos (Data).
\begin{center}
\includegraphics[width=6cm]{Imagenes/img30.png}
\end{center}
5.	En el panel Campos, haga clic en la tabla Clientes, si aún no está seleccionada.\\
\begin{center}
\includegraphics[width=15cm]{Imagenes/img31.png}
\end{center}
6.	Haga clic con el botón derecho en la columna NameStyle y haga clic en Eliminar.\\
\begin{center}
\includegraphics[width=6cm]{Imagenes/img32.png}
\end{center}
7.	En el cuadro de diálogo Eliminar columna, haga clic en Eliminar.\\
\begin{center}
\includegraphics[width=6cm]{Imagenes/img33.png}
\end{center}
8.	Repita los pasos 6 y 7 para la columna SalesPerson.\\
\begin{center}
\includegraphics[width=6cm]{Imagenes/img34.png}
\end{center}
9.	Haga clic con el botón derecho en la columna CustomerID y, a continuación, \\
\begin{center}
\includegraphics[width=6cm]{Imagenes/img35.png}
\end{center}
10.	Haga clic en el encabezado de la columna AddressLine1.\\
11.	En la cinta Modelado, en el grupo Propiedades, haga clic en Categoría de datos: Sin categoría y luego en Dirección.\\
\begin{center}
\includegraphics[width=6cm]{Imagenes/img36.png}
\end{center}
12.	Haga clic en el encabezado de la columna Ciudad.\\
13.	En la cinta Modelado, en el grupo Propiedades, haga clic en Categoría de datos: Sin categoría y luego en Ciudad.\\
\begin{center}
\includegraphics[width=6cm]{Imagenes/img37.png}
\end{center}
14.	Haga clic en el encabezado de la columna StateProvince.\\
15.	En la cinta Modelado, en el grupo Propiedades, haga clic en Categoría de datos: Sin categoría y luego haga clic en Estado o Provincia.\\
\begin{center}
\includegraphics[width=6cm]{Imagenes/img38.png}
\end{center}
16.	Haga clic en el encabezado de la columna CountryRegion.\\
17.	En la cinta Modelado, en el grupo Propiedades, haga clic en Categoría de datos: Sin categoría y luego haga clic en País / Región.\\
\begin{center}
\includegraphics[width=6cm]{Imagenes/img39.png}
\end{center}
18.	Haga clic en el encabezado de la columna PostalCode.\\
19.	En la cinta Modelado, en el grupo Propiedades, haga clic en Categoría de datos: Sin categorizar y luego haga clic en Código postal.\\
\begin{center}
\includegraphics[width=6cm]{Imagenes/img40.png}
\end{center}
20.	En la cinta Modelado, en el grupo Cálculos, haga clic en Nueva columna y luego en la barra de fórmulas, escriba la siguiente expresión y presione Entrar:\\
\begin{center}
\includegraphics[width=12cm]{Imagenes/img41.png}
\end{center}
21. En el panel Campos, haga clic en Ventas.\\
22.	Haga clic con el botón derecho en la columna RevisionNumber y haga clic en Eliminar\\
\begin{center}
\includegraphics[width=6cm]{Imagenes/img42.png}
\end{center}
23.	En el cuadro de diálogo Eliminar columna, haga clic en Eliminar.\\
\begin{center}
\includegraphics[width=6cm]{Imagenes/img43.png}
\end{center}
24.	Realizar el paso 23 y 34 para la columna SalesOrderNumber\\
\begin{center}
\includegraphics[width=6cm]{Imagenes/img44.png}
\end{center}
25.	Haga clic con el botón derecho en la columna CustomerID y luego haga clic en Ocultar en la vista de informe\\
\begin{center}
\includegraphics[width=6cm]{Imagenes/img45.png}
\end{center}
26.	Realizar el paso 25 para las columnas SalesOrderID y SalesOrderDetailID.\\
\begin{center}
\includegraphics[width=6cm]{Imagenes/img48.png}
\end{center}
27.	En la cinta Modelado, en el grupo Cálculos, haga clic en Nueva columna y luego en la barra de fórmulas, escriba la siguiente expresión y presione Entrar:\\
\begin{center}
\includegraphics[width=12cm]{Imagenes/img49.png}
\end{center}
28.	Haga clic en el encabezado de la columna LineTotal.\\
29.	En la cinta Modelado, en el grupo Formato, haga clic en Formato: General, seleccione Moneda y luego haga clic en  Inglés (Estados Unidos).\\
\begin{center}
\includegraphics[width=6cm]{Imagenes/img50.png}
\end{center}
30.	En la cinta Modelado, en el grupo Cálculos, haga clic en Nueva medida y luego en la barra de fórmulas, escriba la siguiente expresión y presione Entrar.\\
\begin{center}
\includegraphics[width=6cm]{Imagenes/img51.png}
\end{center}
31.	Haga clic en Guardar y luego deje Power BI Desktop abierto para la siguiente tarea.\\
\begin{center}
\includegraphics[width=12cm]{Imagenes/img52.png}
\end{center}

\subsubsection*{Tarea 3: Combinar Data\\}

1.	En el Explorador de archivos y luego abra el archivo States.xlsx.\\
\begin{center}
\includegraphics[width=12cm]{Imagenes/img53.png}
\end{center}
2.	En la hoja de trabajo de States, seleccione todos los valores en las dos columnas y luego presione Ctrl + C.\\
3.	En Power BI Desktop, en la cinta Inicio, haga clic en Ingresar datos.\\
\begin{center}
\includegraphics[width=6cm]{Imagenes/img54.png}
\end{center}
4.	En el cuadro de diálogo Crear tabla, haga clic en la tabla y luego presione Ctrl + V. Power BI detecta que la primera fila es un encabezado de columna.\\
5.	En el cuadro "Nombre", escriba "Ventas por estado" y luego haga clic en "Cargar".\\
\begin{center}
\includegraphics[width=12cm]{Imagenes/img55.png}
\end{center}
6.	En la cinta "Inicio", haga clic en Obtener datos y luego en Web.\\
\begin{center}
\includegraphics[width=12cm]{Imagenes/img56.png}
\end{center}
7.	En el cuadro de diálogo Desde Web, en el cuadro URL, escriba\\
\begin{center}
\includegraphics[width=6cm]{Imagenes/img57.png}
\end{center}
8.	En el cuadro de diálogo Navegador, seleccione Códigos y abreviaturas para estados, territorios y otros regiones y luego haga clic en Cargar.\\
\begin{center}
\includegraphics[width=12cm]{Imagenes/img58.png}
\end{center}
9.	En el panel Campos, haga clic en Códigos y abreviaturas de los estados, territorios y otras regiones de EE. UU. mostrar los datos. La tabla tiene 26 filas en la parte inferior que no son necesarias.\\
\begin{center}
\includegraphics[width=5cm]{Imagenes/img59.png}
\end{center}
10.	En la cinta Inicio, en el grupo Datos externos, haga clic en Editar consultas y luego en Editar consultas.\\
\begin{center}
\includegraphics[width=5cm]{Imagenes/img60.png}
\end{center}
11.	En el Editor de consultas, en el panel Consultas, haga clic en Códigos y abreviaturas para estados, territorios y otros regiones.\\
12.	En la cinta Inicio, haga clic en Reducir filas, haga clic en Eliminar filas y, a continuación, haga clic en Eliminar filas inferiores.\\
\begin{center}
\includegraphics[width=12cm]{Imagenes/img61.png}
\end{center}
13.	En el cuadro de diálogo Eliminar filas inferiores, en el cuadro Número de filas, escriba 26 y, a continuación, haga clic en Aceptar.\\
\begin{center}
\includegraphics[width=10cm]{Imagenes/img62.png}
\end{center}
14.	Haga clic en el encabezado de la columna ANSI2 y luego mantenga presionada la tecla Ctrl mientras selecciona todas las columnas a la derecho. Esto selecciona varias filas.\\
\begin{center}
\includegraphics[width=12cm]{Imagenes/img63.png}
\end{center}
15.	Manteniendo presionada la tecla Ctrl, haga clic en las columnas Nombre y estado de region2 y Encabezado para incluirlo en el selección.\\
\begin{center}
\includegraphics[width=12cm]{Imagenes/img64.png}
\end{center}
16.	En la cinta Inicio, haga clic en Administrar columnas, haga clic en Eliminar columnas y luego haga clic en Eliminar columnas.\\
\begin{center}
\includegraphics[width=12cm]{Imagenes/img65.png}
\end{center}
17.	En el panel Configuración de la consulta, en Propiedades, en el cuadro Nombre, escriba Estados con códigos y luego presione Ingresar.\\
\begin{center}
\includegraphics[width=5cm]{Imagenes/img66.png}
\end{center}
18.	En la cinta Inicio, en el grupo Transformar, haga clic en Usar primera fila como encabezados.\\
19.	Haga clic con el botón derecho en el encabezado de la columna Estados Unidos de América, haga clic en Cambiar nombre, escriba Nombre del estado y luego presione Ingresar.\\
\begin{center}
\includegraphics[width=12cm]{Imagenes/img67.png}
\end{center}
20.	Haga clic con el botón derecho en el encabezado de la columna US USA 840, haga clic en Cambiar nombre, escriba Código de estado largo y luego presione Intro.\\
21.	Haga clic con el botón derecho en el encabezado de la columna de EE. UU., Haga clic en Cambiar nombre, escriba Código de estado corto y luego presione Entrar.\\
\begin{center}
\includegraphics[width=12cm]{Imagenes/img68.png}
\end{center}
22.	En el panel Consultas, haga clic en Ventas por estado.\\
23.	En la cinta Inicio, haga clic en Combinar y luego en Combinar consultas.\\
\begin{center}
\includegraphics[width=12cm]{Imagenes/img69.png}
\end{center}
24.	En el cuadro de diálogo Fusionar, en la tabla Ventas por estado, haga clic en la columna Estados.\\
\begin{center}
\includegraphics[width=12cm]{Imagenes/img70.png}
\end{center}
25.	En la lista, haga clic en Estados con códigos, haga clic en la columna Nombre del estado y, a continuación, haga clic en Aceptar. La nueva columna es agregado a la tabla y contiene los estados combinados con la tabla de códigos.\\
26.	En el encabezado de la columna, haga clic en el icono Expandir, desactive (Seleccionar todas las columnas), seleccione Código de estado corto, y luego haga clic en Aceptar. La columna ahora muestra solo los códigos de estado.\\
\begin{center}
\includegraphics[width=12cm]{Imagenes/img71.png}
\end{center}
27.	Haga clic con el botón derecho en la columna, haga clic en Cambiar nombre, escriba Código de estado y luego presione Entrar.\\
28.	En el menú Archivo, haga clic en Cerrar y aplicar.\\
\begin{center}
\includegraphics[width=12cm]{Imagenes/img72.png}
\end{center}
29.	En el panel Campos, haga clic con el botón derecho en Estados con códigos y, a continuación, haga clic en Ocultar en la vista de informe.\\
\begin{center}
\includegraphics[width=12cm]{Imagenes/img73.png}
\end{center}
\begin{center}
\includegraphics[width=12cm]{Imagenes/img74.png}
\end{center}
\begin{center}
\includegraphics[width=12cm]{Imagenes/img75.png}
\end{center}

\hfil \break \hfil \break \hfil \break \hfil \break 
\begin{center}
\section*{ \\Ejercicio 2: Construyendo Reportes en Power BI } 
\end{center}

\subsubsection*{Tarea 1: Crear un Gráfico}
1.	En Power BI Desktop, en la barra derecha de navegación, hacer click en Reporte (Report).\\
2.	En el panel de Visualizaciones (Visualizations), hacer click en Gauge.\\
\begin{center}
\includegraphics[width=5cm]{Imagenes/img77.png}
\end{center}
3.	Arrastar el campo LineTotal de la table Sales a la propiedad Valor (Value) del objeto gauge\\
4.	Arrastrar la medida TargetSales de la table Sales a la propiedad Valor destino (Target value) del objeto gauge.\\
\begin{center}
\includegraphics[width=12cm]{Imagenes/img78.png}
\end{center}
5.	Hacer clic en Formato, expandir Gauge axis, y luego en el cuadro Max, escribir 146000.\\
6.	Expandir Titulo (Title), en el cuadro Texto de Titulo (Title Text), tipear Meta de Ventas (Target Sales), y luego hacer click en Center.\\
\begin{center}
\includegraphics[width=12cm]{Imagenes/img79.png}
\end{center}
7.	Haga clic en el lienzo del informe y luego arrastre el campo CompanyName de la tabla Clientes al informe. Power BI crea automáticamente una tabla.\\
\begin{center}
\includegraphics[width=12cm]{Imagenes/img80.png}
\end{center}
8.	Ponga una estrella en el campo LineTotal de la tabla Sales en el informe.
\begin{center}
\includegraphics[width=5cm]{Imagenes/img81.png}
\end{center}
9.	Asegúrese de que la tabla tenga el foco y, a continuación, en el panel Visualizaciones, haga clic en Gráfico circular.\\
\begin{center}
\includegraphics[width=8cm]{Imagenes/img83.png}
\end{center}
10.	Expanda el gráfico para hacer visibles todos los nombres de las empresas utilizando los controladores de tamaño en el borde del gráfico.\\
11.	Con el foco todavía en el gráfico circular, haga clic en Formato y luego expanda Título.\\
\begin{center}
\includegraphics[width=12cm]{Imagenes/img82.png}
\end{center}
12.	En el cuadro Texto del título, escriba Clientes más vendidos y luego haga clic en Centro.\\
13.	Arrastar el campo MainCategory de la tabla Tabla de ventas en el lienzo del informe. Power BI crea una tabla.\\
\begin{center}
\includegraphics[width=10cm]{Imagenes/img84.png}
\end{center}
14.	Arrastar el campo OrderQty dentro de la tabla.\\
\begin{center}
\includegraphics[width=5cm]{Imagenes/img85.png}
\end{center}
15.	En el panel Visualizaciones, haga clic en Gráfico de barras apiladas.\\
\begin{center}
\includegraphics[width=10cm]{Imagenes/img86.png}
\end{center}
16.	En el panel Visualizaciones, haga clic en Campos.\\
\begin{center}
\includegraphics[width=12cm]{Imagenes/img87.png}
\end{center}
17.	Arrastre el campo OrderQty a la propiedad Color saturation. Observe que los colores cambian.\\
\begin{center}
\includegraphics[width=5cm]{Imagenes/img90.png}
\end{center}
18.	En el panel Visualizaciones, haga clic en Análisis, expanda Línea constante y luego haga clic en Agregar.\\
19.	En el cuadro Valor, escriba 500.\\
\begin{center}
\includegraphics[width=10cm]{Imagenes/img91.png}
\end{center}
20.	Cambie el color a rojo, cambie Etiqueta de datos a Activado y luego cambie el color a rojo.\\
\begin{center}
\includegraphics[width=10cm]{Imagenes/img92.png}
\end{center}
21.	En el panel Visualizaciones, haga clic en Formato y expanda Título.\\
\begin{center}
\includegraphics[width=5cm]{Imagenes/img93.png}
\end{center}
22.	En el cuadro Texto del título, escriba Pedidos por categoría principal y luego haga clic en Centro\\
23.	Haga clic en el lienzo del informe para enfocarlo y, a continuación, en el panel Visualizaciones, haga clic en Gráfico de anillos.\\
\begin{center}
\includegraphics[width=5cm]{Imagenes/img94.png}
\end{center}
24.	En la tabla Sales, seleccione MainCategory y LineTotal.\\
25.	En el panel Visualizaciones, haga clic en Formato y luego expanda Título\\
26.	En el cuadro Texto del título, escriba Ventas por categoría principal y luego haga clic en Centro.\\
27.	Arrastre el campo Producto de la tabla Ventas al lienzo del informe. Power BI crea una tabla\\
28.	Arrastre el campo LineTotal de la tabla Sales al gráfico de la tabla de productos.\\
29.	En la tabla Ventas, seleccione el campo Categoría principal.\\
30.	En el panel Visualizaciones, haga clic en Campos.\\
\begin{center}
\includegraphics[width=12cm]{Imagenes/img95.png}
\end{center}
31.	En el panel Filtros, expanda LineTotal (All).\\
32.	En la lista Mostrar elementos cuando el valor, seleccione es mayor que y, a continuación, en el cuadro siguiente, escriba 32000.\\
33.	Hacer clic en Aplicar filtro (Aplicar filtro).\\
34.	Expanda MainCategory (All) y luego seleccione Bikes.\\
35.	En el panel Visualizaciones, haga clic en Gráfico de columnas apiladas.\\
36.	En el panel Visualizaciones, haga clic en Formato y luego expanda Título.\\\begin{center}
\includegraphics[width=5cm]{Imagenes/img96.png}
\end{center}
37.	En el cuadro Texto del título, escriba Las 10 bicicletas más vendidas y luego haga clic en Centro.\\
38.	En el panel Visualizaciones, haga clic en Análisis, expanda Línea constante y luego haga clic en Agregar\\
\begin{center}
\includegraphics[width=5cm]{Imagenes/img97.png}
\end{center}
39.	En el cuadro Valor, escriba 35000 y luego establezca Color en rojo.\\
40.	Cambie Etiqueta de datos a Activado y, a continuación, establezca Color en rojo\\
\begin{center}
\includegraphics[width=5cm]{Imagenes/img98.png}
\end{center}
41.	Expanda el gráfico para llenar el espacio restante en el lienzo del informe. Si es necesario, mueva sus imágenes para que encajen.\\
42.	Click Save.\\
\begin{center}
\includegraphics[width=5cm]{Imagenes/img99.png}
\end{center}
\subsubsection*{Tarea 2: Crear una Visualización de Mapa}
1.	En la parte inferior del informe, haga clic en el icono + para agregar una nueva página.\\
2.	En el panel Campos, en la tabla Clientes, seleccione el campo Ciudad. Power BI agrega un mapa al informe.\\
\begin{center}
\includegraphics[width=12cm]{Imagenes/img100.png}
\end{center}
3.	En el panel Campos, en la tabla Ventas, seleccione el campo LineTotal.\\
4.	Con la herramienta de captura en el lado derecho del gráfico, cambie el tamaño del mapa para mostrar todas las burbujas.\\
\begin{center}
\includegraphics[width=12cm]{Imagenes/img101.png}
\end{center}
5.	Observe que las burbujas tienen un tamaño proporcional para representar los datos.\\
6.	En el panel Visualizaciones, haga clic en Formato y luego expanda Título.\\
\begin{center}
\includegraphics[width=5cm]{Imagenes/img102.png}
\end{center}
7.	En el cuadro Texto del título, escriba Ventas mundiales por ciudad y luego haga clic en Centro\\
\begin{center}
\includegraphics[width=12cm]{Imagenes/img103.png}
\end{center}
8.	Haga clic en el lienzo del informe y, a continuación, en la tabla Ventas por estado, seleccione la columna Código de estado. Power BI agrega automáticamente un mapa.\\
9.	En la tabla Sales by State, seleccione la columna SalesYTD.\\
\begin{center}
\includegraphics[width=12cm]{Imagenes/img104.png}
\end{center}
10.	En el panel Visualizaciones, haga clic en Mapa lleno. Con la herramienta de captura en el lado derecho y en la parte inferior del gráfico, cambie el tamaño del mapa para mostrar todos los estados.\\
11.	Observe que las ventas se agrupan en un área.\\
\begin{center}
\includegraphics[width=12cm]{Imagenes/img105.png}
\end{center}
12.	Coloque el cursor en California (CA) para ver la cifra de ventas. El valor no se ha formateado como moneda\\
\begin{center}
\includegraphics[width=12cm]{Imagenes/img106.png}
\end{center}
14. En la tabla Sales by State, haga clic en la columna SalesYTD.\\
\begin{center}
\includegraphics[width=12cm]{Imagenes/img107.png}
\end{center}
15.	En la cinta de Modelado, seleccione Formato: General, haga clic en Moneda y luego seleccione  Inglés (Estados Unidos).\\
\begin{center}
\includegraphics[width=12cm]{Imagenes/img108.png}
\end{center}
16.	Coloque el cursor en California (CA) en el mapa y observe que se ha formateado el valor.\\
17.	En el panel Visualizaciones, haga clic en Formato y luego expanda Título\\
\begin{center}
\includegraphics[width=5cm]{Imagenes/img109.png}
\end{center}
18.	En el cuadro Texto del título, escriba Ventas por estado y luego haga clic en Centro.\\
19.	Haga clic en Guardar y luego deje el informe abierto para el próximo ejercicio.\\
\begin{center}
\includegraphics[width=5cm]{Imagenes/img110.png}
\end{center}


\end{document}

